\documentclass{article}

\usepackage{amsmath}
\usepackage[ruled,vlined,linesnumbered]{algorithm2e}
\SetArgSty{textnormal}
\usepackage{hyperref}
\usepackage{natbib}
\usepackage{cleveref}
\crefname{algorithm}{Alg.}{Algs.}
\Crefname{algorithm}{Algorithm}{Algorithms}

\title{COSC-373 Final Report: Bitcoin}
\author{Lee Jiaen, Hyery Yoo, and Alexander Lee}

\begin{document}

\maketitle

\section{Introduction}

Before Bitcoin, e-commerce on the internet relied heavily on trusted third
parties to process electronic payments. For example, suppose Alice, a buyer,
wants to purchase an item from Bob's e-commerce store. Alice would need to make
an online payment that relies on a financial institution or some trusted third
party to process the transaction. However, Alice may not trust such third
parties and wish for a way where her online payment can be sent directly to
Bob without going through a third party. While the previously mentioned payment
system works for most transactions, it has a few significant problems with
efficiency. Trusted third-party financial intuitions cannot avoid mediating
disputes, making it impossible to have completely non-reversible transactions.
Therefore, the need for trust increases, and ``a certain percentage of fraud is
accepted as unavoidable''~\citep{Nakamoto08}. One could avoid these issues with
the use of physical currency. However, like Alice, many people would prefer a
decentralized version of electronic cash that allows any two willing parties to
transact directly without a trusted third party. This is the motivation behind
Bitcoin. This report describes how Bitcoin, an electronic payment system, is
based on cryptographic proof instead of trust, allowing for direct transactions
without trusted third parties or financial institutions.

\section{Preliminaries}

To understand Bitcoin, we first need to develop an understanding of its
fundamental building-blocks: digital signatures and cryptographic hash
functions.

\subsection{Digital Signatures}

Suppose for instance that Alice wants to send a message to Bob over a network.
How can Bob ensure that the message he received was from Alice and not from a
malicious actor like Eve? What Alice can do is append a \emph{digital signature}
to her message and send the message and signature to Bob, allowing Bob to verify
that the message he received was indeed from Alice.

The main idea behind digital signatures is as follows. Both Alice and Bob each
generate a \emph{private key} and \emph{public key} pair for themselves, where
each key is an array of bits and private keys are kept secret (i.e., only Alice
knows her private key and only Bob knows his private key). Producing a signature
involves a function $\mathsf{sign(msg, priKey)}$, which takes Alice's message
$\mathsf{msg}$ and private key $\mathsf{priKey}$ as inputs and outputs Alice's
signature $\mathsf{sig}$. Similarly, verifying the authenticity of the received
message involves a function $\mathsf{verify(msg, sig, pubKey)}$, which takes the
received message $\mathsf{msg}$, the signature $\mathsf{sig}$, and Alice's
public key $\mathsf{pubKey}$ as inputs and outputs \emph{true} if the signature
was produced using Alice's message and private key, and \emph{false} otherwise.
We will not discuss how the $\mathsf{sign()}$ and $\mathsf{verify()}$ functions
work in depth since such details are not the focus of this report.

Digital signatures have two required properties. First, the authenticity of
Alice's signature generated from her message and private key can be verified
easily using her corresponding public key. Secondly, it should be
computationally infeasible for someone like Eve to generate a valid signature
for Alice without knowing Alice's private key. That is, Eve has no better
strategy than guessing and checking if random signatures are valid using Alice's
message and public key. Assuming a 256 bit signature, the aforementioned brute
force strategy would require Eve to check $2^{256}$ signatures in the worst
case.

\subsection{Cryptographic Hash Functions}

A \emph{cryptographic hash function} is a hash function that takes as input data
of an arbitrary size, known as the \emph{message}, and outputs a bit array of a
fixed size, known as the \emph{hash}. In the case of the SHA256 cryptographic
hash function, the hash has a length of 256 bits.

Cryptographic hash functions ideally should have the following properties.
First, computing the hash of a given message should be quick. Second, generating
the message from only the hash should be computationally infeasible. Third,
finding two different messages that map to the same hash under the function
should also be infeasible. Fourth, slightly changing the message should change
the hash so much such that the old and new hashes seem uncorrelated. Similar to
digital signatures, the only way to find a message that produces a given hash is
via a brute force approach of guessing and checking possible messages. Again,
assuming a 256 bit hash, this strategy would require $2^{256}$ hashes to check
in the worst case.

\section{Bitcoin}

The \emph{Bitcoin network} is a randomly connected overlay network of a few
thousand nodes, where nodes are controlled by various owners and perform the
same operation. In other words, the network is a homogeneous network without
central control. Each user of Bitcoin generates a private/public key pair, and a
user is identified by their \emph{address}, which is derived from their public
key and used to receive funds in Bitcoin. All nodes in the Bitcoin network work
together to track each address' balance in bitcoins.

\section{Transactions}

Suppose we have a transaction in bitcoins, where Alice is the sender and Bob is
the recipient. What is a transaction in this case?

A \emph{transaction} is a data structure that describes the transfer of bitcoins
from senders to recipients. The transaction consists of several inputs and
outputs. Each \emph{output} is a tuple composed of an amount in bitcoins and a
spending condition requiring a valid signature associated with the private key
of an address. Meanwhile, each \emph{input} is a tuple consisting of a reference
to a previously created output and a signature to the spending condition,
proving that the transaction creator has the permission to spend the referenced
output. An output has two states, \emph{unspent} and \emph{spent}, and can only
be spent at most once. The sum of the unspent outputs associated with an address
is the \emph{address balance}, and the set of these unspent transaction outputs
(UTXO) is the shared state of Bitcoin. That is, every node in the Bitcoin
network holds a complete replica of the shared state. The local replicas of this
shared state may temporarily diverge, but consistency is eventually
re-established. For a given transaction, the inputs result in the referenced
outputs spent (i.e., removed from the UTXO), and the new outputs are added to
the UTXO.\@ More specifically, inputs reference the output that is being spent
by a $(h, i)$ tuple, where $h$ is the hash of the transaction that created the
output, and $i$ is the index of the output in that transaction.

All transactions are broadcasted to the Bitcoin network and processed by every
node that receives them. Transactions can be in one of two states:
\emph{unconfirmed} or \emph{confirmed}. Incoming transactions are unconfirmed
and added to a pool of transactions referred to the \emph{memory pool}. We will
discuss how transactions are confirmed in \cref{sec:pow}.

Returning to our example, for the transaction between Alice to Bob, the inputs
are Alice's previous unspent outputs and Alice's digital signature, which was
created using her private key. Meanwhile, the outputs of this transaction are
the amount in Bitcoins for Bob and the spending condition requiring a valid
signature associated with the private key of Bob's address. The transaction is
then broadcasted to the Bitcoin network and processed by every node that
receives it.

\section{The Doublespending Problem}

Now suppose there is a transaction from Alice to Charlie that attempts to spend
the same output as the transaction from Alice to Bob. Observe that, for now,
nothing prevents nodes in the Bitcoin network to locally accept different
transactions that spend the same output, i.e., doublespend. More formally,
\emph{doublespending} is a situation in which multiple transactions attempt to
spend the same output. However, only one transaction can be valid because
outputs can only be spent at most once. Doublespending occurs because the order
in which transactions are processed may not be the same for all nodes, and the
validity of transactions depends on the order in which the transactions arrive.
If a node sees two conflicting transactions (i.e., transactions that spend the
same output), the node only considers the first transaction it sees valid and
the second invalid. The second transaction seen is invalid since it tries to
spend an already spent output. As a result, different nodes in the network may
accept different transactions, making their shared states inconsistent. If the
problem of doublespending is not resolved, the shared states of the Bitcoin
network diverge. Therefore, a conflict resolution mechanism is needed to decide
which of the conflicting transactions should be accepted by every node to
achieve eventual consistency in the network.

\section{Processing a Transaction}

Before we continue with discussing the solution to the doublespending problem,
we first summarize how a node processes an incoming transaction (see
\cref{algo:proctrans}). First, the node receives a transaction $t$
(\cref{algline:rectrans}). Then, for each input $(h, i)$ in transaction $t$
(where $h$ is the hash of the transaction that created the referenced output,
and $i$ is the index of the output in that transaction), if $(h, i)$ is not in
the node's local set of unspent transaction outputs or the signature for the
output is invalid (\cref{algline:outputorsig}), then transaction $t$ is dropped
and the node halts. Next, if the sum of the values of inputs is less than the
sum of values of the new outputs (\cref{algline:sums}), then transaction $t$ is
dropped and the node halts since one cannot spend more than they have. Then, for
each input $(h, i)$ in transaction $t$, $(h, i)$ is removed from the node's
local set of unspent transaction outputs (\cref{algline:remove}). Finally,
transaction $t$ is appended to the node's local history of transactions and $t$
is broadcasted to its neighbors in the Bitcoin network.

\begin{algorithm}
  \caption{Process Transaction}\label{algo:proctrans}
  \DontPrintSemicolon%

  Receive transaction $t$\;\label{algline:rectrans}
  \ForEach{input $(h, i)$ in $t$}{
    \If{output $(h, i)$ not in local UTXO \textbf{or} signature
      invalid}{\label{algline:outputorsig}
      Drop $t$ and stop\;
    }
  }
  \If{sum of values of inputs $<$ sum of values of new
    outputs}{\label{algline:sums}
    Drop $t$ and stop\;
  }
  \ForEach{input $(h, i)$ in $t$}{
    Remove $(h, i)$ from local UTXO\;\label{algline:remove}
  }
  Append $t$ to local history\;
  Forward $t$ to neighbors in the Bitcoin network\;
\end{algorithm}

\section{Proof-of-Work}\label{sec:pow}

Bitcoin solves the doublespending problem using \emph{Proof-of-Work} (PoW), a
mechanism in which one party can prove to another that it has spent a specific
amount of computational effort by presenting a nonce $x$ (a bit-string) to the
PoW function $F_d(c,x) \to \{true, false\}$, where $d$ is the difficulty (a
positive number) and $c$ is the challenge (a bit-string). Another party can
confirm this PoW by quickly checking that $F_d(c,x) = true$ for the given $x$.
For instance, suppose that Alice and Bob both know a PoW function $F_d(c,x)$
with $d$ and $c$ given. Alice expends a certain amount of computation to find a
$x$ such that $F_d(c,x) = true$. Alice presents her solution $x$ to Bob, and Bob
checks the solution by computing $F_d(c,x)$ and seeing that the functions output
is $true$. The PoW function must be defined such that verifying a solution $x$
is fast. Through this process, Alice proves to Bob that she has expended the
amount of computational work specified by difficulty $d$.

What does PoW look like in the context of Bitcoin? First, we must understand the
notion of a block in Bitcoin. A \emph{block} is a data structure that consists
of a list of transactions, a reference to a previous block, and a nonce. The
list of transactions are the transactions the node has accepted to its memory
pool since the previous block. Thus, a block encodes incremental changes to the
local state of the node. A node finds a block by finding a valid nonce $x$ that
satisfies the Bitcoin PoW function:
\[
  F_d(c,x) = \text{SHA256}(\text{SHA256}(c|x)) < \frac{2^{224}}{d}.
\]
This process is called \emph{mining}. Simply put, a node must find a value $x$
that, when concatenated with a node-specific challenge $c$ and hashed twice with
the SHA256 hash function, returns a hash that starts with a fixed number of
zeroes. SHA256 is a cryptographic hash function, so a valid nonce can only be
found by iterating over all possible values of $x$, requiring significant
computational work. Once a node finds a valid nonce for the block, the block is
broadcasted to the node's neighbors. Transactions contained in a block are said
to be \emph{confirmed} by that block. The process for finding a block is
summarized in \cref{algo:findblock}.

To make concept of PoW more concrete, suppose Alice is a node who puts in the
computational effort to find a valid nonce $x$ and broadcasts the block
containing $x$ to other nodes. Alice’s PoW can be verified by another node Bob
who sees Alice’s block. Bob is given the values of $c$ and $x$, so he can
quickly compute the hash and verify that the result starts with the required
number of zeroes.

\begin{algorithm}
  \caption{Find Block}\label{algo:findblock}
  \DontPrintSemicolon%

  \SetKwRepeat{Do}{do}{while}
  \KwIn{Nonce $x = 0$, challenge $c$, difficulty $d$, previous block $b_{t-1}$}

  \While{$F_d(c,x) = false$}{
    $x = x + 1$\;
  }
  Broadcast block $b_t = (memory\ pool, b_{t-1}, x)$\;
\end{algorithm}

\section{Receive Block}

A primary goal of the Bitcoin network is for all its nodes to agree on a single
state of the network. To do this, the blocks build a tree with their reference
to a previous block, where the tree's root is known as the \emph{genesis block}.
The longest path from the genesis block to the leaf of the tree is called the
\emph{blockchain}. Eventual consistency is achieved by accepting the longest
path in the tree (i.e., the blockchain) as the true state of the network.
Branches are formed if a node receives two conflicting versions of a block. In
this case, it works on extending the first block it receives and saves the other
block as a branch. If the node later learns that the branch has become longer
than the chain it was working on, it switches to extending the branch. Because
every block requires PoW in order to be added to the chain, the longest chain in
the network represents the greatest amount of collective computational effort,
i.e., majority decision. All nodes eventually receive and accept the longest
chain, so the network reaches consistency. More formally,
\cref{algo:receiveblock} describes how a node updates its local state once it
receive a block. We define the height $h_b$ as the path length from the genesis
block to block $b$. Namely, we denote the node's current head as block $b_{max}$
with height $h_{max}$. First, the node receive block $b$
(\cref{algline:recblock}). Then, the node connects block $b$ in the tree as the
child of its parent $p$ at height $h_b = h_p + 1$. If the height $h_b$ of block
$b$ is larger that the height $h_{max}$ of the node's current head at block
$b_{max}$ (\cref{algline:height}), then the head block and height are updated
appropriately, the node computes the UTXO for the path leading to the new head
block $b_{max}$, and the memory pool is cleaned up.

\begin{algorithm}
  \caption{Receive Block}\label{algo:receiveblock}
  \DontPrintSemicolon%

  \KwIn{Node's current head block $b_{max}$ at height $h_{max}$}

  Receive block $b$\;\label{algline:recblock}
  Connect block $b$ in the tree as child of its parent $p$ at height $h_b = h_p +
  1$\;
  \If{$h_b > h_{max}$}{\label{algline:height}
    $h_{max} = h_b$\;
    $b_{max} = b$\;
    Compute UTXO for the path leading to $b_{max}$\;
    Cleanup memory pool\;
  }
\end{algorithm}

\section{Network}

To summarize, the Bitcoin network operates by each node in the network executing
the same set of protocols. All transactions are broadcasted to the network, and
every node works on collecting new transactions from the network into a block
and finding a valid nonce for the block. When a node finds the valid nonce, it
broadcasts the block to the network, adding the block to the end of the chain in
effect. When another node in the network learns about the new block, it checks
that all transactions in the block are valid. If they are, it accepts the block
and works on extending the chain by creating a new block with the hash of the
accepted block.

\section{Correctness}

Let us now look at how the Bitcoin network solves the doublespending problem.
Suppose Alice attempts to doublespend by creating two transactions containing
the same output, and the two transactions are collected into two separate
blocks, Block 1 and Block 2. If one of the blocks, Block 1, is broadcasted to
the network and gets accepted by all nodes before Block 2 is broadcasted, Block
2 is rejected by all nodes because it contains an invalid transaction. Thus,
Alice’s doublespending attempt fails. In a more subtle case, suppose that Block
1 and Block 2 are broadcasted to the network (almost) simultaneously. Some of
the nodes in the network receive Block 1 first, so they accept Block 1 and
reject Block 2. The opposite is true for other nodes in the network. This
creates two conflicting chains in the network with some of the nodes working on
extending each of the chains. Suppose the majority of nodes accept Block 1. Then
the sum of the computational effort put into Block 1’s chain is greater than
that for Block 2’s chain, and Block 1’s chain grows faster. Eventually, every
node receives this longer chain, and Block 2’s chain becomes rejected by all
nodes, leaving only one of Alice’s transactions in the network.

\section{Conclusion}

This report discusses the fundamental structure of Bitcoin and how it solves the
doublespending problem. Other aspects of Bitcoin include incentivizing miners,
reclaiming unnecessary disk space, and scripting. We refer the reader to learn
more about these topics from the original Bitcoin white paper~\citep{Nakamoto08}
and lecture notes from ETH Zurich by~\citeauthor{ETH15}. Further work aims to
increase the rate of transactions  by establishing micropayment channels between
two parties outside the blockchain network~\citep{Decker15}. Moreover, the
introduction of a decentralized system for financial transactions by Bitcoin led
to a number of other blockchain-based cryptocurrencies such as
Ethereum~\citep{Buterin14}. Bitcoin is just one instance of how distributed
computing can be applied to the field of finance and economics. It would be
interesting to see how else distributed computing can be applied to these
fields.

\bibliographystyle{plainnat}
\bibliography{paper.bib}

\end{document}
