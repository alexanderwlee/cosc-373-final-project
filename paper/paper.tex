\documentclass{article}

\usepackage{hyperref}

\title{COSC-373 Final Report: Bitcoin}
\author{Lee Jiaen, Hyery Yoo, and Alexander Lee}

\begin{document}

\maketitle

\section{Introduction}

\section{Preliminaries}

To understand Bitcoin, we first need to develop an understanding of the
fundamental building-blocks behind it: digital signatures and cryptographic hash
functions.

\subsection{Digital Signatures}

Suppose for example that Alice wants to send a message to Bob over a network.
How can Bob ensure that the message he received was from Alice and not from a
malicious actor like Eve? What Alice can do is append a \emph{digital signature}
to her message and send the message and signature to Bob, which allows Bob to
verify that the message he received was indeed from Alice.

The main idea behind digital signatures is as follows. Both Alice and Bob each
generate a \emph{private key} and \emph{public key} pair for themselves, where
each key is a array of bits and private keys are kept secret (i.e., only Alice
knows her private key and only Bob knows his private key). Producing a signature
involves a function $\mathsf{sign(msg, priKey)}$, which takes Alice's message
$\mathsf{msg}$ and private key $\mathsf{priKey}$ as inputs and outputs Alice's
signature $\mathsf{sig}$. Similarly, verifying the authenticity of the received
message involves a function $\mathsf{verify(msg, sig, pubKey)}$, which takes the
received message $\mathsf{msg}$, the signature $\mathsf{sig}$, and Alice's
public key $\mathsf{pubKey}$ as inputs and outputs \emph{true} if the signature
was produced using Alice's message and private key, and \emph{false} otherwise.
We will not discuss how the $\mathsf{sign()}$ and $\mathsf{verify()}$ functions
work in depth since such details are not the focus of this work.

Digital signatures have two required properties. First, the authenticity of
Alice's signature generated from her message and private key can be verified
easily using her corresponding public key. Secondly, it should be
computationally infeasible for someone like Eve to generate a valid signature
for Alice without knowing Alice's private key. That is, Eve has no better
strategy than guessing and checking if random signatures are valid using Alice's
message and public key. Assuming a 256 bit signature, the aforementioned brute
force strategy would require Eve to check $2^{256}$ signatures in the worst
case.

\subsection{Cryptographic Hash Functions}

A \emph{cryptographic hash function} is a hash function that takes as input data
of an arbitrary size, known as the \emph{message}, and outputs a bit array of a
fixed size, known as the \emph{hash}. In the case of the SHA256 cryptographic
hash function, the hash has a length of 256 bits.

Cryptographic hash functions ideally should have the following properties.
First, computing the hash of a given message should be quick. Second, generating
the message from only the hash should be computationally infeasible. Third,
finding two different messages that map to the same hash under the function
should also be infeasible. Fourth, slightly changing the message should change
the hash so much such that the old and new hashes seem uncorrelated.

\section{Bitcoin}

The \emph{Bitcoin network} is a randomly connected overlay network of a few
thousand nodes, where nodes are controlled by various owners and perform the
same operation. In other words, the network is a homogeneous network without
central control. Each user of Bitcoin generates a private/public key pair. A
user is identified by their \emph{address}, which is derived from their public
key and used to receive funds in Bitcoin. All nodes in the Bitcoin network work
together to track each address' balance in bitcoins.

\section{Transactions}

\section{Doublespends}

\section{Proof-of-Work}

\section{Network}

\section{Conclusion}

\end{document}
