\documentclass{article}

\title{Bitcoin Presentation Notes}
\author{}
\date{}

\begin{document}

\maketitle

\section{Introduction}

\begin{itemize}
  \item Commerce on the Internet relies almost exclusively on financial
    institutions serving as trusted third parties to process electronic
    payments.
  \item Problem: TODO
\end{itemize}

\section{Digital Signatures}

\begin{itemize}
  \item Scenario: Alice wants to send a message to Bob over a network. How can
    Bob verify that the message he received was from Alice? Alice needs to sign
    her message.
  \item Desired property of signature: cannot be forged on a different message.
  \item Both Alice and Bob each generate a public/private key pair for
    themselves. Each key is some string of bits. Private keys are kept secret.
  \item Producing a signature: $\mathsf{sign(message, privateKey)} =
    \mathsf{signature}$.
  \item Verifying a signature: $\mathsf{verify(message, signature, publicKey) =
    true / false}$.
  \item Only the owner of the private key can produce the signature.
  \item No one can copy the signature and forge it on another message.
  \item Signature is a 256 bit value. Hard to find a valid signature if you
    don't know the secret key. There is no strategy better than guessing and
    checking if random signatures are valid using the public key. There are
    $2^{256}$ signatures to check; this is a very large number.
\end{itemize}

\section{Transactions}

\begin{itemize}
  \item Electronic coin: a chain of digital signatures.
  \item Alice transfers a coin to Bob:
    \begin{enumerate}
      \item Alice computes the hash of previous transaction and Bob's public
        key.
      \item Alice signs the hash using her private key.
      \item Alice adds her signature to the end of the coin.
      \item Bob verifies that Alice transferred a coin to him using Alice's
        public key.
    \end{enumerate}
  \item Problem: how to verify that Alice did not double-spend the coin?
\end{itemize}

\end{document}
