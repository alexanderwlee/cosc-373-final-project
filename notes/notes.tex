\documentclass{article}

\title{Bitcoin Presentation Notes}
\author{}
\date{}

\begin{document}

\maketitle

\section{Motivation}

\begin{itemize}
  \item Situation: commerce on the Internet relies almost exclusively on
    financial institutions serving as trusted third parties to process
    electronic payments.
  \item Problem: such third parties may not be trustworthy.
  \item Goal: develop a decentralized version of electronic cash that would
    allow online payments to be sent directly from one party to another without
    going through a financial institution.
\end{itemize}

\section{Digital Signatures}

\begin{itemize}
  \item Scenario: Alice wants to send a message to Bob over a network. How can
    Bob verify that the message he received was from Alice? Alice needs to sign
    her message.
  \item Desired property of signature: cannot be forged on a different message.
  \item Both Alice and Bob each generate a public/private key pair for
    themselves. Each key is some string of bits. Private keys are kept secret.
  \item Producing a signature: $\mathsf{sign(message, privateKey)} =
    \mathsf{signature}$.
  \item Verifying a signature: $\mathsf{verify(message, signature, publicKey) =
    true / false}$.
  \item Only the owner of the private key can produce the signature.
  \item No one can copy the signature and forge it on another message.
  \item Signature is a 256 bit value. Hard to find a valid signature if you
    don't know the secret key. There is no strategy better than guessing and
    checking if random signatures are valid using the public key. There are
    $2^{256}$ signatures to check; this is a very large number.
\end{itemize}

\section{Transactions}

\begin{itemize}
  \item Electronic coin: a chain of digital signatures.
  \item Alice transfers a coin to Bob:
    \begin{enumerate}
      \item Alice computes the hash of the previous transaction and Bob's public
        key. % TODO: maybe describe transaction
      \item Alice signs the hash using her private key.
      \item Alice adds her signature to the end of the coin.
    \end{enumerate}
  \item Bob can verify that Alice transferred a coin to him using Alice's public
    key.
  \item Problem: how to verify that Alice did not double-spend the coin? For
    example, Alice could have transferred a coin to Charlie and then transferred
    the same coin to Bob. For our purposes, the earliest transaction is the one
    that counts, so we should ignore later attempts to double-spend. As of now,
    there is no mechanism preventing Alice from double-spending.
  \item To accomplish this without a trusted party, transactions must be
    publicly announced, and we need a system for participants to agree on a
    single history of the order in which they were received. Bob needs proof
    that at the time of the transaction, the majority of nodes agreed it was the
    first received.
  \item Solution: a distributed timestamp server with proof-of-work.
\end{itemize}

\section{Timestamp Serve}

\begin{itemize}
  \item Take a hash of a block of transactions to be timestamped and widely
    publishing the hash.
  \item The timestamp proves that the data must have existed at the time,
    obviously, in order to get the hash.
  % TODO
\end{itemize}

\section{Cryptographic Hash Functions}

\begin{itemize}
  \item $\mathsf{SHA256(message) = hash}$.
  \item Input: message.
  \item Output: String of 256 bits, known as the hash of the message.
  \item The output looks random even though it isn't (i.e., the function always
    gives the same output for a given input).
  \item Slightly changing the input results in the output to change completely
    in an unpredictable way.
  \item Infusible to compute in the reverse direction. Given only the output,
    finding the input that hashes to the output is extremely difficult. There is
    no better method than to guess and check. Need to go through $2^{256}$
    guesses.
\end{itemize}

\section{Proof-of-Work}

\begin{itemize}
  \item % TODO
\end{itemize}

\section{Network}

\begin{itemize}
  \item The network executes as follows:
    \begin{enumerate}
      \item New transactions are broadcast to all nodes.
      \item Each node collects new transactions into a block.
      \item Each node works on finding a difficult proof-of-work for its block.
      \item When a node finds a proof-of-work, it broadcasts the block to all
        nodes.
      \item Nodes accept the block only if all transactions in it are valid and
        not already spent.
      \item Nodes express their acceptance of the block by working on creating
        the next block in the chain, using the hash of the accepted block as the
        previous hash.
    \end{enumerate}
\end{itemize}

\end{document}
