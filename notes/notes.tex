\documentclass{article}

\title{Bitcoin Presentation Notes}
\author{}
\date{}

\begin{document}

\maketitle

\section{Motivation}

\begin{itemize}
  \item Situation: commerce on the Internet relies almost exclusively on
    financial institutions serving as trusted third parties to process
    electronic payments.
  \item Problem: such third parties may not be trustworthy.
  \item Goal: develop a decentralized version of electronic cash that would
    allow online payments to be sent directly from one party to another without
    going through a financial institution.
  %TODO: add more? How could third parties not be trustworthy?
\end{itemize}

\section{Detour: Digital Signatures}

\begin{itemize}
  \item Scenario: Alice wants to send a message to Bob over a network. How can
    Bob verify that the message he received was from Alice? Alice needs to sign
    her message.
  \item Desired property of signature: cannot be forged on a different message.
  \item Both Alice and Bob each generate a private/public key pair for
    themselves. Each key is some string of bits. Private keys are kept secret.
  \item Producing a signature: $\mathsf{sign(message, privateKey)} =
    \mathsf{signature}$.
  \item Verifying a signature: $\mathsf{verify(message, signature, publicKey) =
    true / false}$.
  \item Only the owner of the private key can produce the signature.
  \item No one can copy the signature and forge it on another message.
  \item Signature is a 256 bit value. Hard to find a valid signature if you
    don't know the secret key. There is no strategy better than guessing and
    checking if random signatures are valid using the public key. There are
    $2^{256}$ signatures to check; this is a very large number.
\end{itemize}

\section{Bitcoin}

\begin{itemize}
  \item Bitcoin network: a randomly connected overlay network of a few thousand
    nodes, controlled by a variety of owners. All nodes perform the same
    operations, i.e., it is homogeneous network and without central control.
  \item Users generate private/public key pairs.
  \item Address: an identifier derived from a user's public key to receive funds
    in Bitcoin.
  \item The Bitcoin network collaboratively tracks the balance in bitcoins of
    each address.
\end{itemize}

\section{Transactions}

\begin{itemize}
  \item Output: a tuple consisting of an amount in bitcoins and a spending
    condition requiring a valid signature associated with the private key of an
    address.
    \begin{itemize}
      \item Outputs exists in two states: unspent and spent.
      \item Any output can be spent at most once.
      \item The address balance is the sum of bitcoin amounts in unspent outputs
        that are associated with the address.
      \item The set of unspent transaction outputs (UTXO) and some additional
        global parameters is the shared state of Bitcoin. Every node in the
        Bitcon network holds a complete replica of that state. Local replicas
        may temporarily diverge but consistency is eventually re-established.
    \end{itemize}
  \item Input: a tuple consisting of a reference to a previously created output
    and a signature to the spending condition, proving that the transaction
    creator has the permission to spend the referenced output.
  \item Transaction: a data structure that describes the transfer of bitcoins
    from spenders to recipients. The transaction consists of a number of inputs
    and new outputs. The inputs result in the referenced outputs spent (removed
    from the UTXO), and the new outputs being added to the UTXO.\@
  \item A transaction from Alice to Bob:
    \begin{itemize}
      \item Sender: Alice.
      \item Recipients: Bob.
      \item Inputs: one or more tuples where each tuple consists of a reference
        to one of Alice's previously created unspent outputs and Alice's
        signature using her private key.
      \item Output: a tuple consisting of the amount in bitcoins for Bob and a
        spending condition requiring a valid signature associated with the
        private key of Bob's address.
    \end{itemize}
  \item Transactions are broadcast in the Bitcoin network and processed by every
    node that receives them.
  \item The outputs of a transaction may assign less than the sum of inputs, in
    which case the difference is called the transaction's fee. The fee is used
    to incentivize other participants in the system.
  \item Transactions are in one of two states: unconfirmed or confirmed. In
    coming transactions from the broad cast are unconfirmed and added to a pool
    of transactions called the memory pool.
\end{itemize}

\section{Doublespends}

\begin{itemize}
  \item Problem: nothing prevents nodes to locally accept different transactions
    that spend the same output. For example, the transaction from Alice to Bob
    could have spent the same output as the transaction from Alice to Charlie.
  \item Doublespend: a situation in which multiple transactions attempt to spend
    the same output. Only one transaction can be valid since outputs can only be
    spent once. When nodes accept different transactions in a doublespend, the
    shared state becomes inconsistent.
  \item Doublespends can result in an inconsistent state since the validity of
    transactions depends on the order in which they arrive. If two conflicting
    transactions are seen by a node, the node considers the first to be valid.
    The second transaction is invalid since it tries to spend an output that is
    already present. The order in which transactions are seen may not be the
    same for all node, hence the inconsistent state.
  \item If doublespends are not resolved, the shared state diverges. Therefore a
    conflict resolution mechanism is needed to decide which of the conflicting
    transactions is to be confirmed (accepted by everybody), to achieve eventual
    consistency.
\end{itemize}

\section{Detour: Cryptographic Hash Functions}

\begin{itemize}
  \item $\mathsf{SHA256(message) = hash}$.
  \item Input: message.
  \item Output: String of 256 bits, known as the hash of the message.
  \item The output looks random even though it isn't (i.e., the function always
    gives the same output for a given input).
  \item Slightly changing the input results in the output to change completely
    in an unpredictable way.
  \item Infeasible to compute in the reverse direction. Given only the output,
    finding the input that hashes to the output is extremely difficult. There is
    no better method than to guess and check. Need to go through $2^{256}$
    guesses.
  \item However, given the input message, the hash is fast to compute.
\end{itemize}

\section{Proof-of-Work and Blocks}

\begin{itemize}
  \item Proof-of-Work (PoW): a mechanism that allows a party to prove to another
    party that a certain amount of computational resource has been utilized for
    a period of time.
  \item Block: a data structure used to communicate incremental changes to the
    local state of a node. A block consists of a list of transactions, a
    reference to a previous block and a nonce. A block lists some transactions
    the block creator (``miner'') has accepted to its memory-pool since the
    previous block. A node finds and broadcasts a block when it finds a valid
    nonce.
  \item A valid nonce is a number such that the SHA256 hash of the block begins
    with a required number of zero bits.
  \item A node finds a valid nonce for a block by incrementing the nonce from
    zero and checking if the SHA256 hash begins with the required number of zero
    bits.
  \item Once a node finds a valid nonce for a block, the block cannot be changed
    without redoing the work. As later blocks are chained after it, the work to
    change the block would include redoing all the blocks after it.
  \item The PoW also solves the problem of determining representation in
    majority decision making. The majority decision is represented by the
    longest chain, which has the greatest PoW effort invested in it.
\end{itemize}

\section{Network}

\begin{itemize}
  \item The network executes as follows:
    \begin{enumerate}
      \item New transactions are broadcast to all nodes.
      \item Each node collects new transactions into a block.
      \item Each node works on finding a difficult proof-of-work for its block.
      \item When a node finds a proof-of-work, it broadcasts the block to all
        nodes.
      \item Nodes accept the block only if all transactions in it are valid and
        not already spent.
      \item Nodes express their acceptance of the block by working on creating
        the next block in the chain, using the hash of the accepted block as the
        previous hash.
    \end{enumerate}
  \item Nodes always consider the longest chain to be the correct one and will
    keep working on extending it.
  \item If two nodes broadcast different versions of the next block
    simultaneously, some nodes may receive one or the other first. In that case,
    they work on the first one they received but save the other branch in case
    it becomes longer; the nodes that were working on the other branch will then
    switch to the longer one.
  \item New transaction broadcasts do not necessarily need to reach all nodes.
    As long as they reach many nodes, they will get into a block before long.
  \item Block broadcasts are also tolerant of dropped messages. If a node does
    not receive a block, it will request it when it receives the next block.
\end{itemize}

\end{document}
